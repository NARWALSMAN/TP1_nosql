% Options for packages loaded elsewhere
\PassOptionsToPackage{unicode}{hyperref}
\PassOptionsToPackage{hyphens}{url}
%
\documentclass[
]{article}
\usepackage{amsmath,amssymb}
\usepackage{iftex}
\ifPDFTeX
  \usepackage[T1]{fontenc}
  \usepackage[utf8]{inputenc}
  \usepackage{textcomp} % provide euro and other symbols
\else % if luatex or xetex
  \usepackage{unicode-math} % this also loads fontspec
  \defaultfontfeatures{Scale=MatchLowercase}
  \defaultfontfeatures[\rmfamily]{Ligatures=TeX,Scale=1}
\fi
\usepackage{lmodern}
\ifPDFTeX\else
  % xetex/luatex font selection
\fi
% Use upquote if available, for straight quotes in verbatim environments
\IfFileExists{upquote.sty}{\usepackage{upquote}}{}
\IfFileExists{microtype.sty}{% use microtype if available
  \usepackage[]{microtype}
  \UseMicrotypeSet[protrusion]{basicmath} % disable protrusion for tt fonts
}{}
\makeatletter
\@ifundefined{KOMAClassName}{% if non-KOMA class
  \IfFileExists{parskip.sty}{%
    \usepackage{parskip}
  }{% else
    \setlength{\parindent}{0pt}
    \setlength{\parskip}{6pt plus 2pt minus 1pt}}
}{% if KOMA class
  \KOMAoptions{parskip=half}}
\makeatother
\usepackage{xcolor}
\usepackage[margin=1in]{geometry}
\usepackage{graphicx}
\makeatletter
\def\maxwidth{\ifdim\Gin@nat@width>\linewidth\linewidth\else\Gin@nat@width\fi}
\def\maxheight{\ifdim\Gin@nat@height>\textheight\textheight\else\Gin@nat@height\fi}
\makeatother
% Scale images if necessary, so that they will not overflow the page
% margins by default, and it is still possible to overwrite the defaults
% using explicit options in \includegraphics[width, height, ...]{}
\setkeys{Gin}{width=\maxwidth,height=\maxheight,keepaspectratio}
% Set default figure placement to htbp
\makeatletter
\def\fps@figure{htbp}
\makeatother
\setlength{\emergencystretch}{3em} % prevent overfull lines
\providecommand{\tightlist}{%
  \setlength{\itemsep}{0pt}\setlength{\parskip}{0pt}}
\setcounter{secnumdepth}{-\maxdimen} % remove section numbering
-\usepackage{color}
\ifLuaTeX
  \usepackage{selnolig}  % disable illegal ligatures
\fi
\IfFileExists{bookmark.sty}{\usepackage{bookmark}}{\usepackage{hyperref}}
\IfFileExists{xurl.sty}{\usepackage{xurl}}{} % add URL line breaks if available
\urlstyle{same}
\hypersetup{
  pdftitle={compte\_rendu\_tp1\_mongo},
  pdfauthor={theo guesdon},
  hidelinks,
  pdfcreator={LaTeX via pandoc}}

\title{compte\_rendu\_tp1\_mongo}
\author{theo guesdon}
\date{2023-10-06}

\begin{document}
\maketitle

\textbackslash begin\{document\}

\section{TP de mongodb}

\subsection{proposition 1}
\paragraph{question}

afficher les titre et la courte description si elle existe(mais pas
obligatoire) des different livres qui on été punlier apres 2008 et qui
sont toujours considéré comme publier et dont le sujet( categorie) parle
d'Internet

\paragraph{resolution}

db.book.find(\{ \$and: {[} \{ publishedDate: \{ \$gt:
ISODate(``2008-12-31T23:59:59.999Z'') \} \}, \{ status: ``PUBLISH'' \}
{]} \}, \{ \_id: 0, title: 1, shortDescription: 1 \})

\paragraph{resultat}

{[} \{ title: `Flex 3 in Action' \},

\{ title: `Flex 4 in Action' \},

\{ title: `Flex on Java', shortDescription: ' A beautifully written book
that is a must have for every Java Developer.Ashish Kulkarni, Technical
Director, E-Business Software Solutions Ltd.' \}, {]}

\subsection{proposition 2}
\paragraph{question}

liste le titre et les autheurs aggrégé regroupper par categories de lire
et trier par date de parution

\paragraph{resolution}

db.books.aggregate({[} \{ \$match: \{ status: ``PUBLISH'',
publishedDate: \{ \$gt: ISODate(``2008-12-31T23:59:59.999Z'') \} \} \},
\{ \(unwind: "\)authors'' \}, \{
\(group: {  _id: { category: "\)categories'', title: ``\$title'' \},
authors: \{ \(push: "\)authors'' \}, publishedDate: \{
\(first: "\)publishedDate'' \} \} \}, \{ \$sort: \{ ``\_id.category'':
1, publishedDate: 1 \} \}, \{ \(group: {  _id: "\)\_id.category'',
books: \{ \(push: {  title:"\)\_id.title'', authors: \{
\(reduce: { input: "\)authors'', initialValue: ````, in: \{
\(concat: ["\)\(value", "\)\$this''{]} \} \} \} \} \} \} \} {]})

\paragraph{resultat}

\{ \_id: {[} `Computer Graphics' {]}, books: {[} \{ title: `Gnuplot in
Action', authors: `Philipp K. Janert' \} {]} \}, \{ \_id: {[} `Internet'
{]}, books: {[} \{ title: `Flex 3 in Action', authors: `Tariq Ahmed with
Jon HirschiFaisal Abid' \}, \{ title: ``Website Owner's Manual'',
authors: `Paul A. Boag' \}, \{ title: `Hello! Flex 4', authors: `Peter
Armstrong' \},

\subsection{proposition 3}
\paragraph{question}

affiche la moyenne du nombre de page des livres avec la categorie
``Internet''

\paragraph{resolution}

db.books.aggregate({[} \{ \$match: \{ categories: ``Internet'', status:
``PUBLISH'', \} \}, \{ \$group: \{ \_id: ``nombre de pages en moyenne'',
averagePageCount: \{ \(avg: "\)pageCount'' \} \} \}{]})

\paragraph{resultat}

{[} \{ \_id: `nombre de pages en moyenne', averagePageCount:
441.2439024390244 \}{]} \textbackslash end\{document\}

\end{document}
